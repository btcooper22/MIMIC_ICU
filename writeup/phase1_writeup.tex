\documentclass[onecolumn]{article}

%Titling
    \usepackage[compact]{titlesec}
    \titlespacing{\section}{0pt}{3ex}{2ex}
    \titlespacing{\subsection}{0pt}{2ex}{1ex}
    \titlespacing{\subsubsection}{0pt}{1ex}{0.5ex}
\titleformat*{\section}{\Large\scshape}
\titleformat*{\subsection}{\Large}
\titleformat{\subsubsection}
  {\normalfont\bfseries}{\thesubsection}{1em}{}
  
%Page size
	\addtolength{\oddsidemargin}{-.875in}
	\addtolength{\evensidemargin}{-.875in}
	\addtolength{\textwidth}{1.75in}

	\addtolength{\topmargin}{-.875in}
	\addtolength{\textheight}{1.75in}
	
	\parskip=.1cm

\usepackage{graphicx}
\graphicspath{ {figures/} }
\usepackage[T1]{fontenc}
\usepackage{lmodern}
\usepackage{float}
\usepackage[table]{xcolor}
\usepackage{multirow}
\usepackage{tabularx} 
\usepackage[table]{xcolor}
\usepackage[font=small,labelfont=bf]{caption}
\usepackage{textcomp}
\usepackage{gensymb}
\usepackage{lineno}
\usepackage{amsmath}
\usepackage{blindtext}
\usepackage{texshade}
\usepackage{bigdelim}
\usepackage{textgreek}
\usepackage{multicol}


%Citations
\usepackage[authoryear]{natbib}
\setcitestyle{authoryear,open={(},close={)}}
\renewcommand{\bibname}{References}

\usepackage[allbordercolors = white, linkcolor = blue, citecolor = blue, colorlinks = true]{hyperref}
\usepackage[nameinlink]{cleveref}

%titling
\title{Modelling Uncertainty in the Risk of Intensive Care Unit Readmission I: Data Extraction and Modelling}
\date{\today}
\author{Ben Cooper}


\begin{document}
\maketitle

\section{Scope and Aims}

This document provides a written overview of the first phase of the project. The aims of this phase were twofold:

\begin{enumerate}
\item To extract a dataset from the MIMIC-III database of surgical ICU patients, consisting of a clearly defined outcome measure (ICU readmission), and a range of predictors.
\item To compare the performance of a range of published models for the prediction of ICU readmission risk and identify the best model to take forward. This will form the prediction model at the core of a system for quantifying uncertainty and dealing with missing data.
\end{enumerate}

\section{Introduction}

\subsection{ICU readmission}

% Brief background on ICU risk stuff

\subsection{Readmission prediction models}

% Overview of structure, strengths and shortcomings of each model

\section{Methods}

% Signpost remaining methods

\subsection{Data Source}

% Overview of MIMIC data

\subsection{Inclusion critera}

% Workflow in extract_patients & preprocess_data

% Flowchart

\subsection{Outcome measure}

% Workflow in define_outcomes

\subsection{Candidate models}

% Outline each model and its predictors

% Overview of exclusion criteria

\subsection{Model comparisons}

% Workflow in compare_scores

\subsection{Recalibration}

% Workflow in recalibrate_models

\subsection{Novel model}

% Workflow for cooper model

% Included variables

\section{Results}

% ROC comparison figure (standard and recalibrated)

% Calibration figure (standard and recalibrated)

% AUROC and HL table (standard and recalibrated)
%			Standard		Recalibrated
%	Model	AUC		HL		AUC		HL

\subsection{Descriptive statistics}

% "Table 1" for each model's variables

% Verbal overview of clear trends here

\subsection{Discrimination}

% Overview of discrimination, before and after recalibration

\subsection{Calibration}

% Overview of discrimination, before and after recalibration

\section{Discussion}

\subsection{Model performance}

% Summarise above, make comparisons

\subsection{Next steps}

% Models to be taken forward

\begin{multicols}{2}
\bibliographystyle{thesis}
{\small
\bibliography{ICU refs}}

\end{multicols}
\end{document}